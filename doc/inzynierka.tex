\documentclass[11pt,a4paper,polish,thesis]{dcsbook}

\usepackage[T1]{fontenc}
\usepackage[utf8]{inputenc}
\usepackage{babel}

\setcounter{secnumdepth}{3}
\setcounter{tocdepth}{3}

\begin{document}

\author{Patryk Dąbrowki 100584\\ Aleksander Kędzierski 98875\\ Paweł Lampe 99277\\ Mateusz Sikora 99615}
\title{Platforma zarządzania zdarzeniami na urządzeniach mobilnych if\{y\}}
\supervisor{dr inż.~Jerzy Błaszczyński}
\date{Poznań, 2014}

\maketitle

\frontmatter

\tableofcontents{}

\mainmatter

\chapter{Wstęp}
\section{Opis problemu i koncepcja jego rozwiązania}
%% poniżej szkic opisu problemu, chyba trzeba go będzie przerobić bo jakoś słabo brzmi
Współczesne urządzenia mobilne dysponują ogromnym zbiorem możliwości. W przypadku chęci złożenia kilku możliwości w pewną usługę, trzeba stworzyć dedykowany kod.
Prowadzi to do zalania rynku aplikacji mobilnych dużą ilością prostych oraz stosunkowo schematycznych implementacji.

Problem można oczywiście rozwiązać tworząc aplikację pozwalającą na preparowanie pewnych schematów używając prostego języka opisu problemu. Ujmując krócej --
można napisać program pozwalający na tworzenie małych programów wykonujących konkretne akcje w kontekście zaistniałych zdarzeń. Obecnie istnieją takie
rozwiązania, jednak są to produkty komercyjne o zamkniętym kodzie.

Koncepcją jaka została przyjęta w niniejszej pracy, to stworzenie otwartoźródłowej biblioteki uproszczające dostęp do podzespołów urządzenia. Celem zwiększenia
atrakcyjności, postanowiono stworzyć także aplikację ukazującą jej możliwości.
\section{Cel i zakres pracy}
Celem niniejszej pracy, jest stworzenie ...
\section{Omówienie pracy}

\chapter{Rozdzialy teoretyczne}
\section{Konkurencyjne rozwiązania}
\subsection{On X}
Aplikacja Microsoftu umożliwiającą kontrolowanie telefonu z Androidem używając kodu w JavaScripcie. Umożliwia wysyłanie Zasad (Rules) na telefon poprzez stronę internetową. Dostęp do funckcjonalości Androida jest zapewniony przez api w postaci Wyzwalaczy (Triggers) i Akcji (Actions). Cały system jest niestety połączony z Facebookiem i wymaga posiadania tam konta.
Na podstawie \cite{onx}.
\subsection{Tasker}
Więcej informacji można znaleźć w książce \cite{sop}.
\chapter{Projekt rozwiązania}
\section{Definicja pojęć}
\begin{itemize}
\item Podfunkcjonalność (ang. Feature) -- Część biblioteki zapewniająca Receptom dostęp do pozdbioru funkcjonalności Androida.
\item Zdarzenie (ang. Event) -- Zmiana stanu systemu, która powoduje uruchomienie kodu Recepty.
\item Recepta (ang. Recipe) -- Napisany przez użytkownika fragment kodu opisujący, co ma się zdarzyć po spełnieniu pewnych warunków.
\item Targowisko (ang. Market) -- Aplikacja internetowa pozwalająca tworzyć i pobierać Recepty.
\item Aplikacja -- Aplikacja androidowa wykorzystująca bibliotekę if\{Y\}. 
\item Serwer Grup -- Komputer z działającą aplikacją, która zarządza grupami użytkowników i Zdarzeniami Grupowymi.
\item Zdarzenie Grupowe -- Zdarzenie związane z Grupą, wysyłane lub odbierane przez Aplikację z Serwera Grup.
\item Grupa -- Zbiór użytkowników identyfikowalny przez nazwę zdefiniowany na Serwerze Grup.
\end{itemize}

\section{Przypadki użycia}
\section{Moduły systemu}
System składa się z biblioteki, przykładowej Aplikacji appIFY oraz aplikacji działających na serwerze - Serwera Grup oraz Targowiska.
Aplikacja korzysta z biblioteki oraz komunikuje się z serwerem. Oprócz tego Serwer Grup oraz Targowisko udostępniają z poziomu przeglądarki takie funkcje jak rejestracja użytkowników czy tworzenie recept.

\chapter{Opis implementacji}
\section{Użyte technologie}
W tej części zaprezentowano opis technologii użytych bezpośrednio w implementacji składowych platformy.
\subsection{Android}
System operacyjny z rodziny Linux przeznaczony dla urządzeń mobilnych. Aktualnie rozwijane przez sojusz biznesowy Open Handset Alliance.
\subsection{Android SDK}
Platforma programistyczna umożliwiająca tworzenie aplikacji dla systemu Android. Zawiera wtyczkę do środowiska Eclipse, narzędzia wspierające prace programisty, emulator i biblioteki potrzebne do zbudowania aplikacji. Programy dedykowne platformie pisane są w języku Java i uruchamiane na maszynie wirtualnej Dalvik.
\subsection{Apache Commons}
\subsection{Apache HTTP Server}
Otwartoźródłowy serwer HTTP. Najpopularniejsze narzędzie tego typu na świecie. Jego wielką zaletą jest mnogość informacji na jego temat dostępnych w internecie oraz
dostępność na większość znaczących systemów operacyjnych.
\subsection{Git}
Rozproszony oraz wieloplatformowy system kontroli wersji będący wolnym oprogramowaniem. Preferowane narzędzie programistów związanych z otwartym oprogramowaniem.
\subsection{HTML 5}
Język programowania służący do tworzenia współczesnych stron internetowych. Jest rozwinięciem oraz uproszczeniem języka HTML 4.
\subsection{Hibernate}
Narzędzie odwzorowań obiektowo-relacyjnych (ang. object-relation mapping, ORM) rozwijany na zasadzie wolnego oprogramowania. Umożliwia odworowania obiektowo-relacyjne, pamięć podręczną, leniwe (ang. Lazy loading), chciwe pobieranie oraz rozproszoną pamięć podręczną.
\subsection{JSON}
Skrót od JavaScript Object Notation. Jest to lekki, tekstowy format wymiany danych niezależny od języka programowania. Został wybrany ze względu na swoją czytelność i wsparcie ze strony bibliotek programistyzcnych.
\subsection{Java 6}
Jezyk programowania cechujący się obiektowością (ang. Object-oriented programming, OOP) oraz silmnym typowaniem. Kod źródłowy Javy kompilowany jest do kodu bajtowego interpretowanego przez maszynę wirtualną zapewnia to większa niezależność od platformy niż w innych podobnych językach np. C++.
\subsection{JavaScript}
Skryptowy język oprogramowania stosowany na stronach internetowych.
\subsection{Apache Maven}
Narzędzie automatycznego budowania oprogramowania dla języka JAVA. Głównymi problemami jakie rozwiązuje Maven przy budowaniu aplikacji są: zarządzanie zależnościami, mozliwość wieloma modułami, wsparcie dla testów.
\subsection{MySQL}
System zarządzania relacyjnymi bazami danych. Jest to wolne oprogramowanie szczególnie upodobane przez twórców aplikacji internetowych. Bardzo dobrze współpracuje z językami takimi jak PHP czy Java
\subsection{PHP}
Obiektowy język programowania dedykowany generowaniu stron internetowych w czasie rzeczywistym. Szczególnie użyteczny w przypadku tworzenia prototypów tudzież niewielkich projektów wymagających stosunkowo niskiego poziomu abstrakcji.
\subsection{RESTeasy}
Framework oprogramowania służacy do tworzenia aplikacji rozproszonych, oparty na wzorcu architektury oprogramowania Representational State Transfer(REST).
\subsection{SpringFramework}
Framework(Szkielet) tworzenia aplikacji w języku Java a w szczególności JavaEE. Do najważniejszych fukcji Springa zalicza się wstrzykiwanie zależności (ang. dependency injection, DI) oraz programowanie aspektowe (ang. aspect-oriented programming, AOP).  
\subsection{Vaadin}
Framework sieciowy służący do tworzenia aplikacji sieciowych w szczególnosci interfejsu użytkownika w oparciu o Google Web Toolkit (GWT) w języku JAVA.


\section{Użyte narzędzia}
\subsection{Apache Tomcat}
Kontener aplikacji siciowych.
\subsection{Eclipse with Android developer tools}
\subsection{String Tool Suite}
Zintegrowane środowisko programistyczne oparte o Eclipsa dostosowany do SpringFramework.
\subsection{Emacs}
Popularny, w pełni rozszerzalny edytor tekstowy spotykany głównie w systemach operacyjnych z rodziny Unix. Używany przez wysokiej klasy programistów oraz naukowców na całym świecie.
\subsection{Git bash for windows}
\subsection{Git for linux}
\subsection{Github}
Serwis internetowy gromadzący społeczność programistów z całego świata. Służy jako hosting dla otwartoźródłowych projektów zarządzanych za pomocą systemu Git.
\subsection{Latex}

\subsection{Linux}
Rodzina systemów operacyjnych będących wolnym oprogramowaniem oraz używajnących jądra Linux.
\subsection{Notepad++}
\subsection{Przeglądarki internetowe}
\subsection{Windows}
\section{Użyty sprzęt}

\subsection{Komputery klasy PC}
\subsection{LG Swift GT540 - }
Procesor: Qualcomm MSM7227 600 MHz
Pamięć RAM: 256 MB
System operacyjny: Android 4.0.1 (Cyanogen mod)
\subsection{Media-Droid IMPERIUS EN3RGY MT7013}
Procesor: dwurdzeniowy, 1GHz ARM7 MTK6577
Pamięć RAM: 256 MB
System operacyjny: Android 4.1.2
\subsection{Motorola Defy MB525}
Procesor: TI OMAP3610 800 MHz
Pamięć RAM: 512 MB
System operacyjny: Android 4.3.1 (Cyanogen mod)
\subsection{Sony LT18 Xperia Arc S}
Procesor: Qualcomm MSM8255T 1,40 GHz
Pamięć RAM: 512 MB
System operacyjny: Android 4.0.4
\subsection{Samsung Galaxy Mini GT-S5570}
Procesor: Qualcomm MSM7227 600 MHz
Pamięć RAM: 384 MB
System operacyjny: Android 2.2
\section{Architektura klienta}
\subsection{Moduł obsługi recept}
\subsection{Moduły dostępu do systemu}

\section{Architektura serwera}
\subsection{Repozytorium recept}
\subsection{Serwer recept grupowych}
% Rozdziały dokumentujące pracę własną studenta (rozdziały opisujące ideę / sposób / metodę 
% rozwiązania postawionego problemu oraz rozdziały opisujące techniczną stronę rozwiązania 
% (dokumentacja techniczna) lub opisujące przeprowadzone testy / badania i uzyskane wyniki)
\section{Opis pakietów}
\subsection{Pakiety Aplikacji}
pl.poznan.put.cs.ify.app - główny pakiet Aplikacji.
pl.poznan.put.cs.ify.jars - pakiet odpowiedzialny za zarządzanie plikami .jar zawierającymii recepty pobrane z Targowiska.
pl.poznan.put.cs.ify.core - pakiet odpowiedzialny za zarządzanie dostępnymi i aktywowanymi Receptami.
pl.poznan.put.cs.ify.appify.receipts - pakiet zawierający Recepty wbudowane w Aplikację.
pl.poznan.put.cs.ify.app.ui - pakiet zawierający kontrolki interfejsu użytkownik.
pl.poznan.put.cs.ify.app.ui.params - pakiet zawierający kontrolki interfejsu użytkownika wykorzystywane do wprowadzania parametrów przy inicjalizacji Recepty.
pl.poznan.put.cs.ify.app.market - pakiet odpowiedzialny za pobieranie danych z Targowiska i wyświetlanie ich.
pl.poznan.put.cs.ify.app.fragments - pakiet zawierający widoki ekranów aplikacji.
\subsection{Pakiety Biblioteki}
pl.poznan.put.cs.ify.api - pakiet główny Biblioteki.
pl.poznan.put.cs.ify.api.exceptions - pakiet zawierający wyjątki, które mogą być rzucane przez metody z Biblioteki.
pl.poznan.put.cs.ify.api.features - pakiet zawietający Podfunkcjonalności i Zdarzenia.
pl.poznan.put.cs.ify.api.group - pakiet odpowiedzialny za obsługę Recept Grupowych.
pl.poznan.put.cs.ify.api.log - pakiet odpowiedzialny za obsługę logowania i domyślny widok logów.
pl.poznan.put.cs.ify.api.params - pakiet zawierający typy parametrów wykorzystywanych przez Recepty.
pl.poznan.put.cs.ify.api.security - pakiet odpowiedzialny za moduł uprawnień Biblioteki.
pl.poznan.put.cs.ify.api.types - pakiet zawierający typy danych wykorzystywanych przez Biblioteke.
\subsection{Pakiety Serwera}
pl.poznan.put.cs.ify.webify - pakiet główny serwera.
pl.poznan.put.cs.ify.webify.data.dao - pakiet zawierający warstwe dostępu do danych.
pl.poznan.put.cs.ify.webify.data.entity - pakiet zawierający klasy odwzorowywane na bazę danych.
pl.poznan.put.cs.ify.webify.data.enums - pakiet zawierajacy potrzebne w bazie danych typy wyliczeniowe(np. lista ról). 
pl.poznan.put.cs.ify.webify.gui - pakiet główny graficznego interfejsu użytkownika.
pl.poznan.put.cs.ify.webify.gui.windows - paiet zawierający wszytskie okna aplikacji sieciowej.
pl.poznan.put.cs.ify.webify.gui.components - pakiet zawierający komponenty użyte w aplikacji.
pl.poznan.put.cs.ify.webify.gui.session - 
pl.poznan.put.cs.ify.webify.service - pakiet zawierający logikę.
pl.poznan.put.cs.ify.webify.rest - pakiet zawerajacy obsługę zapytań typu REST.
pl.poznan.put.cs.ify.webify.utils - pakiet, w którym przechowywane są funkcje pomocnicze używane w całym projkcie.
\chapter{Zakończenie}
% podsumowanie uzyskanych wyników

\appendix

\chapter{Przewodnik użytkownika}
\section{Opis Podfunkcjonalności}
\subsection{Akcelerometr (YAccelerometerFeature.java)}
Umożliwia reagowanie na odczyty akcelerometru wbudowanego w urządzenie. 

\subsection{Battery (YBatteryFeature.java)}
Umożliwia reagowanie na zmiany poziomu baterii urządzenia.

\subsection{SMS (YSMSFeature.java)}
Umożliwia wysyłanie wiadomości SMS oraz reagowanie na wiadomości przychodzące.

\subsection{Wifi (YWifiFeature.java)}
Umożliwia włączanie i wyłączanie modułu WiFi urządzenia.

\subsection{GPS (YGPSFeature.java)}
Umożliwa śledzenie pozycji urządzenia za pomocą modułu GPS.

\subsection{Sound (YSoundFeature.java)}

\subsection{RawPlayer (YRawPlayerFeature.java)}

\subsection{Group (YGroupFeature.java)}

\subsection{Geocoder (YGeocoderFeature.java)}
Umożliwia pobranie adresu związanego z podaną długościa i szerokością geograficzną.

\subsection{Time (YTimeFeature.java)}

\subsection{AudioManager (YAudioManager.java)}

\subsection{Text (YTextFeature.java)}

\subsection{Internet (YInternetFeature.java)}
Umożliwia wysyłanie i pobieranie danych z podanego adresu.

\subsection{Calls (YCallsFeature.java)}
Umożliwia reagowanie na połączenia przychodzące i inicjowanie połączeń wychodzących.

\subsection{Notification (YNotificationFeature.java)}
Umożliwia wyświetlanie powiadomień w interfejsie graficznym urządzenia.

\backmatter

\begin{thebibliography}{1}
\bibitem{onx}Projekt on\{X\} http://www.onx.ms/\#!findOutMorePage. Ostatnio odwiedzone 6/02/13.
\bibitem{sop}A.~Tanenbaum. \emph{Operating Systems Design and Implementation}. Prentice Hall, 2006.
\bibitem{springinaction}C.~Walls. \emph{Spring in action, 3rd edition}. Manning Publication Co, 2011.
\bibitem{vaadinbook}Vaadin https://vaadin.com/book/vaadin6/-/page/preface.html 
\bibitem{patterns}E.~Gamma. \emph{Design Patterns, First edition}. Person Education, Inc, 1995.
\end{thebibliography}

\end{document}
