\documentclass[11pt,a4paper,polish,thesis]{dcsbook}

\usepackage[T1]{fontenc}
\usepackage[utf8]{inputenc}
\usepackage{babel}

\setcounter{secnumdepth}{3}
\setcounter{tocdepth}{3}

\begin{document}

\author{Patryk Dąbrowki Aleksander Kędzierski\\ Paweł Lampe Mateusz Sikora}
\title{Tytuł pracy}
\supervisor{dr inż.~Jerzy Błaszczyński}
\date{Poznań, 2014}

\maketitle

\frontmatter

\tableofcontents{}

\mainmatter

\chapter{Wstęp}
Wprowadzenie do tematu...

\section*{Cel i zakres pracy}
Celem niniejszej pracy jest...

\chapter{Rozdzialy teoretyczne}
% przegląd literatury naświetlający stan wiedzy na dany temat
Więcej informacji można znaleźć w książce \cite{sop}.

\chapter{Rozdzialy dokumentujące pracę studenta}
% Rozdziały dokumentujące pracę własną studenta (rozdziały opisujące ideę / sposób / metodę 
% rozwiązania postawionego problemu oraz rozdziały opisujące techniczną stronę rozwiązania 
% (dokumentacja techniczna) lub opisujące przeprowadzone testy / badania i uzyskane wyniki)

\chapter{Zakończenie}
% podsumowanie uzyskanych wyników

\appendix

\chapter{Przewodnik użytkownika}
% wat ?

\backmatter

\begin{thebibliography}{1}
\bibitem{sop}A.~Tanenbaum. \emph{Operating Systems Design and Implementation}. Prentice Hall, 2006.
\end{thebibliography}

\end{document}
