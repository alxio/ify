\documentclass[11pt,a4paper,polish,thesis]{dcsbook}

\usepackage[T1]{fontenc}
\usepackage[utf8]{inputenc}
\usepackage{babel}

\setcounter{secnumdepth}{3}
\setcounter{tocdepth}{3}

\begin{document}

\author{Patryk Dąbrowki Aleksander Kędzierski\\ Paweł Lampe Mateusz Sikora}
\title{Tytuł pracy}
\supervisor{dr inż.~Jerzy Błaszczyński}
\date{Poznań, 2014}

\maketitle

\frontmatter

\tableofcontents{}

\mainmatter

\chapter{Wstęp}
Wprowadzenie do tematu...
\section{Opis problemu i koncepcja jego rozwiązania}
\section{Cel i zakres pracy}
\section{Omówienie pracy}

\chapter{Rozdzialy teoretyczne}
% przegląd literatury naświetlający stan wiedzy na dany temat
\section{Konkurencyjne rozwiązania}
Więcej informacji można znaleźć w książce \cite{sop}.

\chapter{Projekt rozwiązania}
\section{Definicja pojęć}
\section{Przypadki użycia}
\section{Moduły systemu}
\section{Architektura systemu}


\chapter{Opis implementacji}
\section{Użyte technologie}
\subsection{Android}
System operacyjny z rodziny Linux przeznaczony dla urządzeń mobilnych. Aktualnie rozwijane przez sojusz biznesowy Open Handset Alliance.
\subsection{Android SDK}
Platforma programistyczna umożliwiająca tworzenie aplikacji dla systemu Android. Zawiera wtyczkę do środowiska Eclipse, narzędzia wspierające prace programisty, emulator i biblioteki potrzebne do zbudowania aplikacji. Programy dedykowne platformie pisane są w języku Java i uruchamiane na maszynie wirtualnej Dalvik.
\subsection{Apache Commons}
\subsection{Apache Server}
\subsection{Git}
Rozproszony system kontroli wersji. Zapewnia wersjonowanie plików, umożliwa wielu członkom zespołu pracę nad jednym projektem.
\subsection{HTML 5}
\subsection{Hibernate}
\subsection{JSON}
Skrót od JavaScript Object Notation. Jest to lekki, tekstowy format wymiany danych niezależny od języka programowania. Został wybrany ze względu na swoją czytelność i wsparcie ze strony bibliotek programistyzcnych.
\subsection{Java}
\subsection{JavaScript}
\subsection{Maven}
\subsection{MySQL}
\subsection{PHP}
\subsection{REST Easy}
\subsection{Spring}
\subsection{Vaadin}

\section{Użyte narzędzia}
\subsection{Apache Tomcat}
\subsection{Eclipse with Android developer tools}
\subsection{Eclipse with String Tool Suite}
\subsection{Emacs}
\subsection{Git bash for windows}
\subsection{Git for linux}
\subsection{Github}
\subsection{Latex}
\subsection{Linux}
\subsection{Notepad++}
\subsection{Przeglądarki internetowe}
\subsection{Windows}
\section{Użyty sprzęt}
\subsection{Komputery klasy PC}
\subsection{LG Swift GT540 - Cyanogen based on Android 4.0.1}
\subsection{Media-Droid IMPERIUS EN3RGY MT7013 - Android 4.1.2}
\subsection{Motorola Defy MB525 - Cyanogen based on Android 4.3.1}
\subsection{Sony Xperia Arc S Lti18 - Android 4.0.4}

\section{Architektura klienta}
\subsection{Moduł recept}
\subsection{Moduły dostępu do systemu}

\section{Architektura serwera}
\subsection{Repozytorium recept}
\subsection{Serwer recept grupowych}
% Rozdziały dokumentujące pracę własną studenta (rozdziały opisujące ideę / sposób / metodę 
% rozwiązania postawionego problemu oraz rozdziały opisujące techniczną stronę rozwiązania 
% (dokumentacja techniczna) lub opisujące przeprowadzone testy / badania i uzyskane wyniki)

\chapter{Zakończenie}
% podsumowanie uzyskanych wyników

\appendix

\chapter{Przewodnik użytkownika}
% wat ?

\backmatter

\begin{thebibliography}{1}
\bibitem{sop}A.~Tanenbaum. \emph{Operating Systems Design and Implementation}. Prentice Hall, 2006.
\end{thebibliography}

\end{document}
