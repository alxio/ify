\documentclass[11pt,a4paper,polish,thesis]{dcsbook}

\usepackage[T1]{fontenc}
\usepackage[utf8]{inputenc}
\usepackage{babel}

\setcounter{secnumdepth}{3}
\setcounter{tocdepth}{3}

\begin{document}

\author{Patryk Dąbrowki 100584\\ Aleksander Kędzierski 98875\\ Paweł Lampe 99277\\ Mateusz Sikora 99615}
\title{Platforma zarządzania zdarzeniami na urządzeniach mobilnych if\{y\}}
\supervisor{dr inż.~Jerzy Błaszczyński}
\date{Poznań, 2014}

\maketitle

\frontmatter

\tableofcontents{}

\mainmatter

\chapter{Wstęp}
Wprowadzenie do tematu...
\section{Opis problemu i koncepcja jego rozwiązania}
\section{Cel i zakres pracy}
\section{Omówienie pracy}

\chapter{Rozdzialy teoretyczne}
% przegląd literatury naświetlający stan wiedzy na dany temat
\section{Konkurencyjne rozwiązania}
\subsection{On X}
Aplikacja Microsoftu umożliwiającą kontrolowanie telefonu z Androidem używając kodu w JavaScripcie. Umożliwia wysyłanie Zasad (Rules) na telefon poprzez stronę internetową. Dostęp do funckcjonalości Androida jest zapewniony przez api w postaci Wyzwalaczy (Triggers) i Akcji (Actions). Cały system jest niestety połączony z Facebookiem i wymaga posiadania tam konta.
Na podstawie \cite{onx}.
\subsection{Tasker}
Więcej informacji można znaleźć w książce \cite{sop}.
\chapter{Projekt rozwiązania}
\section{Definicja pojęć}
\begin{itemize}
\item Podfunkcjonalność (ang. Feature) -- Część biblioteki zapewniająca Receptom dostęp do pozdbioru funkcjonalności Androida.
\item Zdarzenie (ang. Event) -- Zmiana stanu systemu, która powoduje uruchomienie kodu Recepty.
\item Recepta (ang. Recipe) -- Napisany przez użytkownika fragment kodu opisujący, co ma się zdarzyć po spełnieniu pewnych warunków.
\item Targowisko (ang. Market) -- Aplikacja internetowa pozwalająca tworzyć i pobierać Recepty.
\item Aplikacja -- Aplikacja androidowa wykorzystująca bibliotekę if\{Y\}. 
\item Serwer Grup -- Komputer z działającą aplikacją, która zarządza grupami użytkowników i Zdarzeniami Grupowymi.
\item Zdarzenie Grupowe -- Zdarzenie związane z Grupą, wysyłane lub odbierane przez Aplikację z Serwera Grup.
\item Grupa -- Zbiór użytkowników identyfikowalny przez nazwę zdefiniowany na Serwerze Grup.
\end{itemize}

\section{Przypadki użycia}
\section{Moduły systemu}
\section{Architektura systemu}


\chapter{Opis implementacji}
\section{Użyte technologie}
W tej części zaprezentowano opis technologii użytych bezpośrednio w implementacji składowych platformy.
\subsection{Android}
\subsection{Apache Commons}
\subsection{Apache Server}
\subsection{Git} 
\subsection{HTML 5}
\subsection{Hibernate}
Narzędzie odwzorowań obiektowo-relacyjnych (ang. object-relation mapping, ORM) rozwijany na zasadzie wolnego oprogramowania. Umożliwia odworowania obiektowo-relacyjne, pamięć podręczną, leniwe ładowanie(ang. Lezy loading),chciwe pobieranie oraz rozproszoną pamięć podręczną.
\subsection{JSON}
JavaScript Object Notation, lekki format danych wywodzący się z języka JavaScript(jest jego podzbiorem). 
\subsection{Java 6}
\subsection{JavaScript}
\subsection{Apache Maven}
Narzędzie automatycznego budowania oprogramowania dla języka JAVA. Głównymi problemami jakie rozwiązuje Maven przy budowaniu aplikacji są: zarządzanie zależnościami, mozliwość wieloma modułami, wsparcie dla testów.
\subsection{MySQL}
System zarządzania relacyjnymi bazami danych. Jest to wolne oprogramowanie szczególnie upodobane przez twórców aplikacji internetowych. Bardzo dobrze współpracuje z językami takimi jak PHP czy Java
\subsection{PHP}
Obiektowy język programowania dedykowany generowaniu stron internetowych w czasie rzeczywistym. Szczególnie użyteczny w przypadku tworzenia prototypów tudzież niewielkich projektów wymagających stosunkowo niskiego poziomu abstrakcji.
\subsection{RESTeasy}
Framework oprogramowania służacy do tworzenia aplikacji rozproszonych, oparty na wzorcu architektury oprogramowania Representational State Transfer(REST).
\subsection{SpringFramework}
Framework(Szkielet) tworzenia aplikacji w języku Java a w szczególności JavaEE. Do najważniejszych fukcji Springa zalicza się wstrzykiwanie zależności (ang. dependency injection, DI) oraz programowanie aspektowe (ang. aspect-oriented programming, AOP).  
\subsection{Vaadin}
Framework sieciowy służący do tworzenia aplikacji sieciowych w szczególnosci interfejsu użytkownika w oparciu o Google Web Toolkit (GWT) w języku JAVA.


\section{Użyte narzędzia}
\subsection{Apache Tomcat}
\subsection{Eclipse with Android developer tools}
\subsection{String Tool Suite}
Zintegrowane środowisko programistyczne oparte o Ecplipsa dostosowany do SpringFramework.
\subsection{Emacs}
Popularny, w pełni rozszerzalny edytor tekstowy spotykany głównie w systemach operacyjnych z rodziny Unix. Używany przez wysokiej klasy programistów oraz naukowców na całym świecie.
\subsection{Git bash for windows}
\subsection{Git for linux}
\subsection{Github}
\subsection{Latex}
\subsection{Linux}
\subsection{Notepad++}
\subsection{Przeglądarki internetowe}
\subsection{Windows}
\section{Użyty sprzęt}

\subsection{Komputery klasy PC}
\subsection{LG Swift GT540 - Cyanogen based on Android 4.0.1}
\subsection{Media-Droid IMPERIUS EN3RGY MT7013 - Android 4.1.2}
\subsection{Motorola Defy MB525 - Cyanogen based on Android 4.3.1}
\subsection{Sony Xperia Arc S Lti18 - Android 4.0.4}

\section{Architektura klienta}
\subsection{Moduł obsługi recept}
\subsection{Moduły dostępu do systemu}

\section{Architektura serwera}
\subsection{Repozytorium recept}
\subsection{Serwer recept grupowych}
% Rozdziały dokumentujące pracę własną studenta (rozdziały opisujące ideę / sposób / metodę 
% rozwiązania postawionego problemu oraz rozdziały opisujące techniczną stronę rozwiązania 
% (dokumentacja techniczna) lub opisujące przeprowadzone testy / badania i uzyskane wyniki)

\chapter{Zakończenie}
% podsumowanie uzyskanych wyników

\appendix

\chapter{Przewodnik użytkownika}
% wat ?

\backmatter

\begin{thebibliography}{1}
\bibitem{onx}Projekt on\{X\} http://www.onx.ms/\#!findOutMorePage. Ostatnio odwiedzone 6/02/13.
\bibitem{sop}A.~Tanenbaum. \emph{Operating Systems Design and Implementation}. Prentice Hall, 2006.
\bibitem{springinaction}C.~Walls. \emph{Spring in action, 3rd edition}. Manning Publication Co, 2011.
\bibitem{vaadinbook}Vaadin https://vaadin.com/book/vaadin6/-/page/preface.html 
\bibitem{patterns}E.~Gamma. \emph{Design Patterns, First edition}. Person Education, Inc, 1995.
\end{thebibliography}

\end{document}
